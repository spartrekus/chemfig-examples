
\documentclass[11pt]{article}
\usepackage[utf8]{inputenc}
\usepackage{chemfig}

\usepackage{url}
\usepackage{hyperref}
\usepackage{graphicx}
\usepackage{grffile}
\usepackage{pdfpages}
\usepackage{wallpaper}

\usepackage{epstopdf}
\usepackage{enumitem}

\usepackage{soul}

\sloppy
\usepackage[none]{hyphenat}

\usepackage[utf8]{inputenc}


\begin{document}

have a look at : 
http://www.nagel-net.de/Latex/DOKU/Chemie\_chemfig.pdf\vspace{1cm}

\chemfig{-[:30]=[:-30,,,,red]-[:30]}
\vspace{1cm}


\chemfig{C(-[2]H)(-[4]H)(-[6]H)-\lewis{260,Cl}}
\vspace{1cm}

\chemfig{[:40]H-\lewis{13,O}-[::-80]H}
\vspace{1cm}


\lewis{0.2.4.6.,C}\hspace{1cm}
\vspace{1cm}

\lewis{0|2.46.,C}\hspace{1cm}
\vspace{1cm}

\lewis{1357,Ar}
\vspace{1cm}


\chemfig{**6(------)} \hspace{.5cm} + \hspace{.5cm} \chemfig{H_3C-Cl}
\hspace{.5cm} $\Rightarrow$ \hspace{.5cm}
\chemfig{**6(---(-)---)} \hspace{.5cm} + \hspace{.5cm} \chemfig{H-Cl}


\end{document}
